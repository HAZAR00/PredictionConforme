\documentclass[a4paper,11pt]{article}%classe du document: rapport, article, livre,...

% Packages
\usepackage[T1]{fontenc}
\usepackage[utf8]{inputenc}
\usepackage[french]{babel} %pour écrire en français et utiliser les accents facilement.
\usepackage{amsmath}%pour ecrire des maths
\usepackage{amssymb}%pour utiliser des symboles spécifiques
\usepackage{graphicx}%package pour manipuler des graphiques
\usepackage{hyperref}
\usepackage{color}

% Title and Author
\title{Definitions}
\author{Hazar HAMOUDA - Mohamed MEGDICHE}
\date{\today}

\begin{document}

\maketitle

\section{Fonction score $s$} 


On definit tout d'abord les objets mathematiques qui  nous seront utile dans l'ennonce du theoreme de la prediction conforme : 

Une \emph{fonction de score} (ou fonction de non-conformité) est une application
\[
s : \mathcal{X} \times \mathcal{Y} \rightarrow \mathbb{R}
\]
qui mesure à quel point une paire $(x, y)$ est atypique ou en désaccord avec les données d'entraînement ou le modèle prédictif.

Un score élevé $s(x, y)$ indique que la valeur $y$ est peu plausible compte tenu de $x$, selon le modèle ou une heuristique choisie. Cette fonction est utilisée pour comparer de nouveaux exemples avec les exemples de calibration, indépendamment de la distribution sous-jacente.
\section{ Le quantile $\hat{q}$}
     avec $\hat{q}$ comme le quantile d'ordre $\left\lceil \frac{(n+1)(1 - \alpha)}{n} \right\rceil$ des scores de calibration $s_1 = s(X_1, Y_1), \dots, s_n = s(X_n, Y_n)$.
\section{Ensemble de prédiction conforme $\hat{C}_n(x)$}


    avec : 
    \[
    \hat{C}_n : \mathcal{X} \rightarrow \left\{ \text{sous-ensembles de } \mathcal{Y} \right\}
    \]
    qui associe à chaque entrée $x \in \mathcal{X}$ un ensemble $\hat{C}_n(x) \subseteq \mathcal{Y}$ représentant les valeurs plausibles de sortie $y$.
    
    Cet ensemble est défini par :
    \[
    \hat{C}_n(x) = \left\{ y \in \mathcal{Y} : s(x, y) \leq \hat{q} \right\}
    \]
    
    Autrement dit, pour une nouvelle observation $x$, on considère toutes les sorties $y$ dont le score est inférieur ou égal au seuil $\hat{q}$, déterminé à partir des données de calibration. 
    

\end{document}