\documentclass{beamer}

% Packages
\usepackage[T1]{fontenc}
\usepackage[utf8]{inputenc}
\usepackage[french]{babel}
\usepackage{amsmath}
\usepackage{amssymb}
\usepackage{graphicx}
\usepackage{hyperref}

% Title Information
\title{Introduction au \LaTeX}
\author{M. Bailly-Bechet}
\institute{Université Claude Bernard Lyon 1 \\ Laboratoire de Biométrie et Biologie Évolutive}
\date{\today}

\begin{document}

% Title slide
\frame{\titlepage}

% Slide: Introduction
\begin{frame}{Introduction}
\TeX{} est un logiciel d’édition développé par Donald Knuth, puis modifié par Leslie Lamport (\LaTeX{}), permettant de produire des documents de qualité digne de la publication professionnelle.
\end{frame}

% Slide: LaTeX vs WYSIWYG
\begin{frame}{\LaTeX{} vs WYSIWYG}
\begin{itemize}
    \item \LaTeX{} est un logiciel libre.
    \item Le formatage est semi-automatisé, permettant de se concentrer sur le contenu.
    \item Qualité typographique professionnelle.
\end{itemize}
\end{frame}

% Slide: Les fichiers LaTeX
\begin{frame}{Les fichiers \LaTeX}
\begin{itemize}
    \item \texttt{.tex} : Fichier source avec les commandes.
    \item \texttt{.dvi} : Résultat de la compilation standard.
    \item \texttt{.pdf} : Destiné à la publication après conversion.
\end{itemize}
\end{frame}

% Slide: Document minimal
\begin{frame}[fragile]{Document minimal}
\begin{verbatim}
\documentclass{article}
\begin{document}
Tout ce que je veux afficher dans mon document
\end{document}
\end{verbatim}
\end{frame}

% Slide: Mathématiques
\begin{frame}{Mathématiques}
\begin{block}{Modes mathématiques}
\begin{itemize}
    \item En ligne : \texttt{\$...\$} ou \texttt{\(...\)}
    \item Centré : \texttt{\[...\]} ou \texttt{\$\$...\$\$}
\end{itemize}
\end{block}

\begin{block}{Exemple}
Soit $x$ une variable réelle solution de l’équation :
\[
ax^2 + bx + c = 0
\]
Le discriminant vaut $\Delta = b^2 - 4ac$. S’il est strictement positif, il y a deux racines réelles distinctes :
\[
x_1 = \frac{-b - \sqrt{\Delta}}{2a}, \quad x_2 = \frac{-b + \sqrt{\Delta}}{2a}
\]
\end{block}
\end{frame}

% Slide: Exercices avancés
\begin{frame}{Exercices avancés}
\begin{enumerate}
    \item Navier-Stokes :
    \[
    \frac{\partial \vec{v}}{\partial t} + \left( \vec{v} \cdot \nabla \right) \vec{v} = -\frac{1}{\rho} \nabla p + \nu \nabla^2 \vec{v} + \vec{f}
    \]
    \item Lotka-Volterra :
    \[
    \frac{dx}{dt} = x(\alpha - \beta y), \quad \frac{dy}{dt} = -y(\gamma - \delta x)
    \]
    \item Intégrale gaussienne :
    \[
    \delta \int_{0}^{\infty} \int_{0}^{\infty} e^{-(x^2 + y^2)} \, dx \, dy = \frac{\pi}{4}
    \]
\end{enumerate}
\end{frame}

% Slide: Bibliographie
\begin{frame}[fragile]{Bibliographie}
Exemple de fichier \texttt{.bib}:
\begin{verbatim}
@BOOK{HofbSigm98,
title = {Evolutionary Games and Population Dynamics},
publisher = {Cambridge University Press},
year = {1998},
author = {Joseph Hofbauer, Karl Sigmund}
}
\end{verbatim}
\end{frame}

% Slide: Conclusion
\begin{frame}{Conclusion}
Pour aller plus loin : \url{http://www.jalix.org/ressources/miscellaneous/tex/_faq-latex2/html/}
\end{frame}

\end{document}
