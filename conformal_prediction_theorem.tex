\documentclass[a4paper,12pt]{article}%classe du document: rapport, article, livre,...

% Packages
\usepackage[T1]{fontenc}
\usepackage[utf8]{inputenc}
\usepackage[french]{babel} %pour écrire en français et utiliser les accents facilement.
\usepackage{amsmath}%pour ecrire des maths
\usepackage{amssymb}%pour utiliser des symboles spécifiques
\usepackage{graphicx}%package pour manipuler des graphiques
\usepackage{hyperref}
\usepackage{color}

% Title and Author
\title{Prediction conforme}
\author{Hazar HAMOUDA - Mohamed MEGDICHE}
\date{\today}

\begin{document}

\maketitle

\section{Introduction}
Le but de ce document est démontrer le théorème de la prédiction conforme.
\section{Prédiction Conforme}
Théorème :Soient : 
 * (Xi, Yi)i=1,...,n et (Xtest, Ytest), variables aléatoires idientiques et independentes.
 *
 * q le ⌈(n+1)(1−α)⌉quantile des scores de calibration s1 = s(X1, Y1), ..., sn = s(Xn, Yn)
\section{Conclusion}
deux fcts qui prennent en entrée les models et ensemble de test ( 1 ere calcule la taille de l'intervalle moyen( pour la regression et la cardinalite pour  la classification)(Éeme clacule empiriquement que ds 1-alpha des cas on a bien ce quii est prédit est ds l'intervalle ))
on utilise  np.quantile ( il faut utiliser la bonne method(regarder dcoumentation) )
et il faut appliquer prediction conforme sur le model ( o caclcule scores sur calibration)
intervel x+_quantile et l'autre on prend ceux dui soft max en aqequation avec le quantile
\end{document}
