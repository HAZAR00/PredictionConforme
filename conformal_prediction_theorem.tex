\documentclass[a4paper,12pt]{article}%classe du document: rapport, article, livre,...

% Packages
\usepackage[T1]{fontenc}
\usepackage[utf8]{inputenc}
\usepackage[french]{babel} %pour écrire en français et utiliser les accents facilement.
\usepackage{amsmath}%pour ecrire des maths
\usepackage{amssymb}%pour utiliser des symboles spécifiques
\usepackage{graphicx}%package pour manipuler des graphiques
\usepackage{hyperref}
\usepackage{color}

% Title and Author
\title{Prediction conforme}
\author{Hazar HAMOUDA - Mohamed MEGDICHE}
\date{\today}

\begin{document}

\maketitle

\section{Introduction}
Le but de ce document est démontrer le théorème de la prédiction conforme.
\section{Prédiction Conforme}
Théorème :Soient : 
 * (Xi, Yi)i=1,...,n et (Xtest, Ytest), variables aléatoires idientiques et independentes.
 *
 * q le ⌈(n+1)(1−α)⌉quantile des scores de calibration s1 = s(X1, Y1), ..., sn = s(Xn, Yn)
\section{Conclusion}
\end{document}
